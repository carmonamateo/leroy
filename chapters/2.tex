%%%%%%%%%%%%%%%%%%%%%%%%%%%%%%%%%%%%%%%%%%%%%%%%%%%%%%%%%%%%%%%
\chapter*{\S \space II. --- OBJETS LOCALEMENT CONSTANTS ET OBJETS GALOISIENS}\thispagestyle{empty}
\label{sec:2}
\section*{}

On se donne un $\cU$-topos $E$.

{\bf 2.1. Objets localement constants}

\vskip .3cm
{
Définitions {\bf 2.1.1.} --- \it
\begin{enumerate}
    \item[1)] Soient $A$ et $B$ deux objets de $E$. On dit que $A$ trivialise $B$ si $A \times B \xlongrightarrow{pr_1} A$ est un objet constant du topos $E_{/A}$.
    \item[2)] Nous dirons qu'un objet $L$ de $E$ est \emph{localement constant} si les objets de $E$ qui trivialisent $L$ recouvrent $E$.
    \item[3)] Enfin nous dirons qu'un préfaisceau $F$ sur une catégorie $C$ est localement constant si pour tout morphisme $X \to Y$ de $C$, l'application $F(Y) \to F(X)$ est bijective.
\end{enumerate}
}

\vskip .3cm
{
Proposition {\bf 2.1.2.} --- \it Soient $Y$ un objet de $E$ et $D \xlongrightarrow{u} C$ un morphisme de $E$. 
\begin{enumerate}
    \item[(i)] Si $C$ trivialise $Y$, $D$ trivialise $Y$ ;
    \item[(ii)] si, en outre, $C$ et $D$ sont connexes et non-vides, l'application $f \mapsto f \circ u$ de $\Hom(C, Y)$ dans $\Hom(D, Y)$ est \emph{bijective}.
\end{enumerate}
}
\vskip .3cm

{\it Démonstration}. Prenons un ensemble $I$ et un $C$-isomorphisme $\alpha: C \times Y \isom I_C$.
\begin{enumerate}
    \item[(i)] On a des diagrammes cartésiens
    $$
        \xymatrix{
            D\times Y \ar[r] \ar[d] & C\times Y \ar[d]\\
            D \ar[r] & C
        }
        \qquad \qquad 
        \xymatrix{
            I_D \ar[r] \ar[d] & I_C \ar[d]\\
            D \ar[r] & C
        }
    $$
    d'où un $D$-isomorphisme
    $$
    \beta: D \times Y \to I_D.
    $$
    \item[(ii)] Soient $f: C \to Y$ et $g: D \to Y$. Pour que $f \circ u = g$, il faut et il suffit que le diagramme
    $$
    \xymatrix{
        D\times Y \ar[r] & C\times Y\\
        D \ar[u]^{(1_D,g)} \ar[r] & C \ar[u]_{(1_C,f)}
    }
    $$
    soit commutatif, ou encore que soit commutatif le diagramme
    $$
        \xymatrix{
            I_D \ar[r] & I_C \\
            D \ar[r] \ar[u]^{\beta\circ (1_D,g)}
            & C \ar[u]_{\alpha\circ (1_C,f)}
        }
    $$
    d'où le point $(ii)$ par (1.3).
\end{enumerate}

\vskip .3cm
{
Proposition {\bf 2.1.3.} --- \it 
\begin{enumerate}
    \item[(i)] Soient $C$ et $X$ deux objets de $E$. On a un morphisme naturel
    $$
    p: \Hom(C, X)_C \to X
    $$
    c'est le morphisme qui, pour tout $f \in \Hom(C, X)$, rend commutatif le diagramme
    $$
        \xymatrix{
            C \ar[d]_{i_f} \ar[dr]^f \\
            \Hom(C, X)_C \ar[r]_-p & X
        }
    $$
    où $i_f$ désigne la section de $\Hom(C, X)_C$ au-dessus de $C$ définie par $f$.

    On a donc aussi un $C$-morphisme naturel :
    $$
    m = (q, p): \Hom(C, X)_C \to C \times X
    $$
    où $q$ désigne le $C$-morphisme $\Hom(C, X)_C \to C$.
    \item[(ii)] Si $C$ est connexe non-vide et trivialise $X$, le $C$-morphisme naturel
    $$
    \Hom(C, X)_C \to C \times X
    $$
    est un \emph{isomorphisme}.
\end{enumerate}
}
\vskip .3cm

{\it Démonstration}. En effet, prenons un ensemble $I$ et un $C$-isomorphisme $I_C \to C \times X$. On en tire une bijection :
$$
\Hom_C(C, I_C) \isom \Hom_C(C, C \times X) \isom \Hom(C, X)
$$
d'où un diagramme de $E_{/C}$~:
$$
    \xymatrix{
        (\Hom_C(C, I_C))_C \ar[r]^-\sim \ar[d] & \Hom(C, X)_C \ar[d]\\
        I_C \ar[r] & C\times X
    }
$$
dont la commutativité prouve notre assertion.

\vskip .3cm
{
Proposition {\bf 2.1.4.} --- \it Soient $p$ un point de $E$, $C$ un objet connexe de $E$, $y_0$ un point de la fibre $p^{-1}(C)$ et $X$ un objet de $E$. Si $C$ trivialise $X$, l'application
\begin{align*}
\Hom(C, X) & \to p^{-1}(X)\\
f & \mapsto f(y_0)
\end{align*}
est \emph{bijective}.
}
\vskip .3cm

{\it Démonstration}. En effet, elle se déduit de l'application
\begin{align*}
p^{-1}(C) \times\Hom(C, X) & \to p^{-1}(C) \times p^{-1}(X))\\
(y, f) & \mapsto (y, f(y))
\end{align*}
qui provient, par passage aux fibres, de l'isomorphisme naturel
$$
\Hom(C, X)_C \to C \times X \quad (2.1.3, ii)
$$

\vskip .3cm
{
Proposition {\bf 2.1.5.} --- \it Soient $L$ un objet localement constant de $E$ et $U$ l'image de $L \to e_E$. Il existe un objet $V$ de $E$ tel que $U \amalg V$ soit isomorphe à $e_E$.
}
\vskip .3cm

{\it Démonstration}. Recouvrons $e_E$ par des objets $(U_i)_{i \in I}$ qui trivialisent $L$. Soit $I_0$ (resp. $I_1$) l'ensemble des $i \in I$ tels que $U_i \times L$ soit vide (resp. non-vide). Pour tout $i \in I_0$ et tout $j \in I_1$, $U_i \times U_j$ est vide; en effet, il existe un ensemble non-vide $F$ tel que $F_{U_i \times U_j}$ soit vide.

Soient $V_0$ et $V_1$ respectivement les images de :
$$
\amalg_{i \in I_0} U_i \to e_E, \quad \amalg_{i \in I_1} U_i \to e_E.
$$
Évidemment, $V_1$ est un sous-objet de $U$, et $U \times V_0$ est vide. Comme $e_E$ est somme de $V_0$ et $V_1$, on conclut que $V_1 \isom U$ et $U \amalg V_0 \isom e_E$. 

\vskip .3cm
{\bf 2.2.} Nous supposons maintenant le topos $E$ \emph{localement connexe}

\vskip .3cm
{
Proposition {\bf 2.2.1.} --- \it Soient $Z$ un objet de $E$ et $S$ une sous-catégorie génératrice de $E$ dont les objets sont connexes et non-vides dans $E$. Les propriétés suivantes sont équivalentes :
\begin{enumerate}
    \item[a)] Tout objet de $S$ trivialise $Z$ ;
    \item[b)] Pour tout morphisme $D \to C$ de $S$, l'application correspondante $\Hom(C, Z) \to \Hom(D, Z)$ est bijective ; autrement dit, $\Hom(-, Z)$ est un préfaisceau localement constant sur $S$.
\end{enumerate}
}
\vskip .3cm

{\it Démonstration}. (a) $\Rightarrow$ (b) : c'est (2.1.2). (b) $\Rightarrow$ (a) : soit $C_0$ un objet de $S$. Pour tout objet $D$ de $S$, l'application
\begin{align*}
\Hom(D, C_0) \times \Hom(C_0, Z) & \to \Hom(D, C_0) \times \Hom(D, Z)\\
(u, f) & \mapsto (u, f \circ u)
\end{align*}
au-dessus de $\Hom(D, C_0)$ est bijective, d'où un $C_0$-isomorphisme
$$
\Hom(C_0, Z)_{C_0} \to C_0 \times Z.
$$

\vskip .3cm
{
Proposition {\bf 2.2.2.} --- \it Soit $S$ une sous-catégorie génératrice de $E$ dont les objets sont connexes et non-vides dans $E$. Munissons $S$ de la topologie induite par $E$. Tout préfaisceau localement constant sur $S$ est un faisceau.
}
\vskip .3cm

{\it Démonstration}. Soit $F$ un préfaisceau localement constant sur $S$. Il est clair que $F$ est un préfaisceau séparé ; prouvons que c'est un faisceau.

Soient $U$ un objet de $S$ et $R$ un crible couvrant $U$ (i.e. $R$ contient une famille épimorphique de $E$) ; soit enfin une section
$$
t: R \to F.
$$
Pour tout objet $V$ de $S$ et toute $f: V \to R$, désignons par $x(V, f)$ la section de $F$ au-dessus de $U$, image inverse de $t(f) \in F(V)$ par la bijection $F(U) \to F(V)$ correspondant à $f: V \to U$. Étant donné un diagramme commutatif
$$
\xymatrix{
    V_0 \ar[d] \ar[dr]^{f_0}\\
    V_1 \ar[r]_{f_1} & R
}
$$
on a $x(V_0, f_0) = x(V_1, f_1)$ ; donc (d'après (1.3)) les $x(V, f)$ sont tous égaux à un même $x \in F(U)$ ; et on a $x_{/R} = t$ par construction de $x$.

\vskip .3cm
{
Définition auxiliaire {\bf 2.2.3.} --- \it
\begin{enumerate}
    \item[(i)] Soit $L$ un objet localement constant de $E$. Soit $S$ une sous-catégorie génératrice de $E$, dont les objets sont connexes non-vides dans $E$ et trivialisent $L$. On appellera \emph{site générateur adapté à} $L$ une telle sous-catégorie, munie de la topologie induite par $E$. 
    \item[(ii)] Si $(L_i)_{1 \leq i \leq n}$ est une famille finie d'objets localement constants de $E$, il existe un site générateur de $E$ adapté à chacun des $L_i$ : cela découle de (2.1.2, (i)) par récurrence sur $n$.
\end{enumerate}
}
\vskip .3cm

\vskip .3cm
{
Proposition {\bf 2.2.4.} ($\varprojlim$ et $\varinjlim$ d'objets localement constants). --- \it

Soient $I$ une petite catégorie et 
$L: I \to E$
un foncteur tel que $L(i)$ soit localement constant pour tout $i \in \Ob(I)$. \emph{Supposons qu'il existe un site générateur $S$ adapté à tous les $L(i)$} ; alors :
\begin{enumerate}
    \item[(i)] $P = \varprojlim\limits_{I} L(i)$ est localement constant et $S$ est un site générateur adapté à P (2.2.1).
    \item[(ii)] Le préfaisceau sur $S$ :
    $$
    C \to L(C) = \varinjlim_I \Hom (C, L(i))
    $$
    est un faisceau ; donc $\varinjlim\limits_{I} L(i)$ est un objet localement constant, isomorphe à $L$ en tant que faisceau sur $S$ (en particulier, $S$ est un site générateur adapté à $\varinjlim L(i)$).
    \item[(iii)] Corollaire : Soit $u: L \to M$ un morphisme entre objets localement constants de $E$, et soit $S$ un site générateur adapté à $L$ et $M$. En tant que faisceau sur $S$, l'image de $u$ est isomorphe à 
    $$
    C \to \Img(\Hom(C, L) \to \Hom(C, M)).
    $$
\end{enumerate}
}
\vskip .3cm

\vskip .3cm
{
Proposition {\bf 2.2.5.} --- \it Soit $L$ un objet localement constant de $E$.
\begin{enumerate}
    \item[(i)] Tout sous-objet localement constant $K$ de $L$ est somme directe de composantes connexes de $L$.
    \item[(ii)] Soient $C$ un objet connexe non-vide de $E$ et $K$ un sous-objet de $L$ qui est somme directe de composantes connexes de $L$. Si $C$ trivialise $L$, $C$ trivialise $K$.
\end{enumerate}
}
\vskip .3cm

{\it Démonstration}.
\begin{enumerate}
    \item[(i)] Soit $S$ un site générateur adapté à $K$ et à $L$. Le préfaisceau sur $S$
    $$
    C \to M(C) = \Hom(C, L)\ - \ \Hom(C, K)
    $$
    est un faisceau (2.2.2), et $L$ est somme directe de $K$ de $M$, c.q.f.d.
    \item[(ii)] Si $L$ est somme directe de deux sous-objets $K$ et $K'$, alors $C \times L$ qui est constant dans $E_{/C}$ est somme directe de $C \times K$ et $C \times K'$, qui sont donc constants dans $E_{/C}$ puisque $C$ est connexe et non-vide.
\end{enumerate}

\vskip .3cm
{
Corollaires {\bf 2.2.6.} --- \it 
\begin{enumerate}
    \item[(a)] Tout morphisme d'un objet localement constant non-vide de $E$ dans un objet localement constant connexe de $E$ est un épimorphisme.
    \item[(b)] Soient des morphismes de $E$ :
    $$
    \xymatrix{
        L \ar@<1ex>[r]^f \ar@<-1ex>[r]_g & M
    }
    $$
    où $L$ et $M$ sont localement constants et $L$ en outre, connexe. Si le noyau de $(f, g)$ est non-vide, alors $f = g$.
\end{enumerate}
}
\vskip .3cm

((a) découle de 2.2.4, (iii) et (b) de 2.2.4, (i)).

\vskip .3cm
{\bf 2.3. Objets galoisiens}

\emph{On ne suppose plus} le topos $E$ localement connexe.

\vskip .3cm
{\bf 2.3.1.} {Définition. --- \it Nous dirons qu'un objet $Y$ de $E$ est \emph{galoisien} s'il est localement constant, connexe et non-vide, et s'il est un pseudo-torseur sous le groupe constant $\Aut(Y)_E$.
}
\vskip .3cm
{\it Remarques}.
\begin{enumerate}
    \item[(i)] D'après (2.1.5.), cela revient à dire que l'image $U$ de $Y \to e_E$ est une composante connexe de $e_E$ (1.1), que $Y$ est connexe, et qu'il est un torseur, dans $E_{/U}$, sous le groupe constant $\Aut(Y)_U$.
    \item[(ii)] Tout objet galoisien se trivialise lui-même.
\end{enumerate}

\vskip .3cm
{
Proposition {\bf 2.3.2.} --- \it Soient $A$ et $B$ deux objets connexes de $E$ tels que $A \times B$ soit non-vide. Si $A$ et $B$ se trivialisent l'un l'autre, ils sont isomorphes.
}
\vskip .3cm

{\it Démonstration}. En effet, il existe alors des ensembles non-vides $I$ et $J$ tels que $I_A$ et $J_B$ soient tous deux isomorphes à $A \times B$. Puisque les composantes connexes de $I_A$ (resp. $J_B$) sont toutes isomorphes à $A$ (resp. $B$), $A$ et $B$ sont isomorphes.

\vskip .3cm
{
Corollaire {\bf 2.3.3.} --- \it 
Soient $A$ et $B$ deux objets galoisiens de $E$. Si $\Hom(A, B)$ et $\Hom(B, A)$ sont non-vides, alors $A$ et $B$ sont isomorphes.
}

\vskip .3cm
{
Proposition {\bf 2.3.4.} --- \it Soient $A$ et $B$ deux objets galoisiens de $E$. Si $A$ et $B$ sont isomorphes, tout morphisme de $A$ dans $B$ est un isomorphisme.
}
\vskip .3cm

{\it Démonstration}. Il suffit de prouver que tout morphisme $A \to A$ est un automorphisme. Or (2.1.3.) le $A$-morphisme canonique
$$
\Hom(A, A)_A \to A \times A
$$
est un isomorphisme, et sa restriction à $\Aut(A)_A$ est un isomorphisme puisque $A$ est un pseudo-torseur sous $\Aut(A)_A$. Donc
$$
\Hom(A, A) = \Aut(A).
$$

\vskip .3cm
{\bf 2.3.5.} Soient $G$ un groupe, $B^G$ le topos des $G$--ensembles à droite et $T^G \in \Ob(B^G)$ l'ensemble $G$ muni de l'opération de $G$ par translations à droite.

L'opération de $G$ sur l'ensemble $G$ par translations à gauche fait de $T^G$ un torseur de $B^G$ sous le groupe constant $G_{B^G}$. Pour tout morphisme de topos $E \xrightarrow{u} B^G$, on a donc une structure de $G_E$-torseur à gauche sur $u^{-1}(T^G)$ ; d'où un foncteur
$$
\Homtop(E, B^G)^\circ \to \Tors(E, G_E) \leqno{\text{(*)}}
$$
de la catégorie opposée à $\Homtop(E, B^G)$ dans la catégorie des $G_E$-torseurs à gauche de $E$.
\vskip .3cm
{ 
Lemme. --- \it Le foncteur (*) est une équivalence de catégories.
}
\vskip .3cm
{\it Démonstration abrégée}.
\begin{enumerate}
    \item[a)] Le foncteur (*) est pleinement fidèle parce que $\{T^G\}$ engendre $B^G$.
    \item[b)] Pour tout objet $F$ de $B^G$, l'opération de $G$ sur l'ensemble $F$ fournit une opération du groupe constant $G_E$ sur l'objet constant $F_E$.
\end{enumerate}
Étant donné un $G_E$-torseur à gauche $T$, le foncteur
\begin{align*}
u^{-1}_T: B^G & \to E \\
F & \mapsto T \wedge_{G_E} F_E
\end{align*}
(produit contracté) définit un morphisme de topos $u_T: E \to B^G$ et $u^{-1}_T(T^G)$ est isomorphe à $T$ en tant que $G_E$-torseur.

\vskip .3cm
{\bf 2.3.6.} (``Théorie de Galois'')

Nous supposons le topos $E$ connexe.

Soient $Y$ un objet galoisien de $E$ et $G$ le groupe des automorphismes de $Y$.

On a une structure de $G_E$-torseur à gauche sur $Y$ (2.3.1., remarque (i)), d'où, suivant 2.3.5., un morphisme de topos
$$
u: E \to B^G
$$
tel que le $G_E$-torseur $u^{-1}(T^G)$ soir isomorphe au $G_E$-torseur $Y$.

Le foncteur image directe $u_*$ est isomorphe au foncteur
$$
\phi: E \to B^G
$$
qui associe à l'objet $X$ de $E$ l'ensemble
$$
\phi(X) = \Hom(Y, X)
$$
muni de l'opération à droite de $G = \Aut(Y)$ par composition (en effet, tout faisceau $F$ sur $B^G$ est canoniquement représenté par l'ensemble $F(T^G)$ muni de l'opération à droite de $G$ déduite de l'opération de $G$ sur l'objet $T^G$).

Soit $\LC(E, Y)$ la sous-catégorie pleine de $E$ formée des objets qui sont trivialisés par $Y$ (N. B. ces objets sont localement constants puisque $Y$ recouvre l'objet final). Nous allons prouver que
\begin{enumerate}
    \item[(i)] Le foncteur image inverse
    $$
    u^{-1}: B^G \to E
    $$
    est pleinement fidèle et prend ses valeurs dans $\LC(E, Y)$.
    \item[(ii)] La restriction à $\LC(E, Y)$ du foncteur image directe $u_*\simeq\phi$ est pleinement fidèle.
    D'où les corollaires~:
    \item[(iii)] Les foncteurs $\phi_{/\LC(E, Y)}$ et $u^{-1}$ fournissent des équivalences quasi-inverses
    $$
    \xymatrix{ \LC(E,Y) \ar@<-1ex>[r]  & \ar@<-1ex>[l] B^G}
    $$
    \item[(iv)] La catégorie $\LC(E, Y)$ est un $\cU$-topos et l'inclusion $\LC(E, Y) \to E$ définit un morphisme de topos $E \to \LC(E, Y)$.
\end{enumerate}

{\it Démonstration}.
\begin{enumerate}
    \item[(i)] $T^G$ trivialise tous les objets de $B^G$; donc, étant donné un objet $F$ de $B^G$, $u^{-1}(T^G) \isom Y$ trivialise $u^{-1}(F)$. Prouvons maintenant que le morphisme canonique
    $$
    F \to \phi u^{-1}(F)
    $$
    donné par l'adjonction entre $\phi$ et $u^{-1}$ est un isomorphisme (ce qui entraîne la pleine fidélité de $u^{-1}$).
    \begin{enumerate}
        \item[a)] Le morphisme
        $$
        T^G \to \phi u^{-1}(T^G)
        $$
        est un isomorphisme. En effet, $\phi u^{-1}(T^G)$ est un $G$-torseur puisque $u^{-1}(T^G)$ est isomorphe à $Y$ (2.3.4).
        \item[b)] $\phi$ commute aux sommes directes. (1.2)
        \item[c)] Puisque $T^G$ trivialise $F$, le morphisme
        $$
        T^G \times F \to \phi u^{-1}(T^G \times F) \isom \phi u^{-1}(T^G) \times \phi u^{-1}(F)
        $$
        est un isomorphisme par (a) et (b) ; donc il en est de même du morphisme
        $$
        F \to \phi u^{-1}(F)
        $$
        puisque $T^G$ recouvre l'objet final de $B^G$.
    \end{enumerate}
    \item[(ii)] Soit $L$ un objet de $\LC(E, Y)$. Le $Y$-isomorphisme :
    $$
    \phi(L)_Y = \Hom(Y, L)_Y \isommap Y \times L \leqno{(2.1.3.)}
    $$
    montre qu'on obtient pour tout objet $M$ de $E$ une bijection : 
    $$
    \Hom(Y \times L, M) \isommap \App(\phi(L), \phi(M))
    $$
    en associant au morphisme $m: Y \times L \to M$ l'application
    \begin{align*}
    \phi(L) &\to \phi(M)\\
    f & \mapsto m(f) = m \circ (1_Y, f).
    \end{align*}
    Or on a le diagramme commutatif
    $$
    \xymatrix{
        \Hom(L,M) \ar@{_{(}->}[d] \ar[dr] \\
        \Hom(Y\times L,M) \ar[r]^-\sim & \App(\phi(L), \phi(M))
    }
    $$
    (où la flèche verticale désigne la composition avec la projection $Y \times L \to L$). L'application
    $$
    \Hom(L, M) \to \App(\phi(L), \phi(M))
    $$
    est donc injective ; il reste seulement à prouver que son image est formée des applications qui sont de morphismes de $G$-ensembles.
\end{enumerate}

\vskip .3cm
{\bf 2.4.} Nous supposons à nouveau le topos $E$ localement connexe. Et nous %f
faisons l'hypothèse suivante sur l'univers $\cU$ :

$\cU$ admet un élément de cardinal infini.

Désignons par $\SLC(E)$ la sous-catégorie pleine de $E$ formée des objets qui sont sommes directes d'objets localement constants (i.e. des objets dont les composantes connexes sont des objets localement constants 2.2.5). Les points 2.4.1. à 2.4.10 qui suivent vont prouver le

\vskip .3cm
{
Théorème {\bf 2.4.} --- \it 
\begin{enumerate}
    \item[(i)] La catégorie $\SLC(E)$ est un $\cU$-topos, et l'inclusion $\SLC(E) \to E$ définit un morphisme de topos $E \to \SLC(E)$.
    \item[(ii)] Les objets galoisiens de $\SLC(E)$, qui s'identifient à ceux de $E$, engendrent le topos $\SLC(E)$.
\end{enumerate}
}

\vskip .3cm
{
Proposition {\bf 2.4.1.} --- \it
\begin{enumerate}
    \item[(i)] $\SLC(E)$ est stable dans $E$ par sommes directes et limites projectives finies.
    \item[(ii)] Étant donnés un objet $A$ de $\SLC(E)$ et une relation d'équivalence $R \hookrightarrow A \times A$ de $E$, le quotient $A/R$ est dans $\SLC(E)$ dès que $R$ y est. 
\end{enumerate}
}
\vskip .3cm

Compte tenu de (2.2.4), (2.2.5) et (2.2.6, a), cela découle du lemme suivant :

\vskip .3cm
{
Lemme {\bf 2.4.2.} --- \it Soient $T$ un $\cU$-topos, $K$ une sous-catégorie pleine de $T$, $S$ la sous-catégorie pleine de $T$ formée des objets qui sont sommes directes d'objets de $K$. Supposons que:
\begin{enumerate}
    \item[1)] Tout objet de $K$ est somme directe d'objets connexes $\in \Ob(K)$ ; et tout sous-objet d'un objet $X$ de $K$ qui est somme directe de composantes connexes de $X$ est dans $K$.
    \item[2)] Tout morphisme d'un objet non-vide $\in \Ob(K)$ dans un objet connexe $\in \Ob(K)$ est un épimorphisme (dans $T$).
    \item[3)] $K$ est stable dans $T$ par limites projectives finies.
    \item[4)] Pour tout objet $X$ de $K$ et toute relation d'équivalence $R \hookrightarrow X \times X$ telle que $R \in \Ob(K)$, le quotient $X_{|_R}$ est dans $K$.
\end{enumerate}
Alors :
\begin{enumerate}
    \item[(i)] $S$ est stable dans $E$ par sommes directes et par limites projectives finies.
    \item[(ii)] Pour tout objet $A$ de $S$ et toute relation d'équivalence $R \hookrightarrow A \times A$ telle que $R \in \Ob(S)$, le quotient $A_{|_R}$ est dans $S$.
\end{enumerate}
}
\vskip .3cm

{\it Démonstration}. Le point (i) est immédiat ; prouvons (ii). Nous considérons un diagramme cartésien et cocartésien de $T$
$$
    \xymatrix{
        R \ar[r]^{p_1} \ar[d]_{p_2} & A \ar[d]^\pi\\
        A \ar[r]_\pi & Q
    }
$$
où $A$ et $R$ sont objets de $S$. D'après (1), $A$ est somme directe d'objets connexes $\in \Ob(K)$
$$
\amalg_{i \in I} C_i \isommap A.
$$
Pour tout couple d'indices $(i, j)$, soit
$$
R_{i j} = C_i \times_Q C_j.
$$
Prouvons que $R_{i j}$ est objet de $K$. $R_{i j}$ est produit fibré de $R$ avec $C_i \times C_j$ au-dessus de $A \times A$ ; puisque ces trois objets sont dans $S$, $R_{i j}$ est dans $S$ ; donc $R_{i j} \hookrightarrow C_i \times C_j$ est somme directe de composantes connexes de $C_i \times C_j$, qui est dans $K$ ; donc par (1), $R_{i j}$ est dans $K$.

Pour tout $i \in I$, soit $M_i \hookrightarrow Q$ l'image de $\pi_{|_{C_i}}$. On a $C_i \times_{M_i} C_i = C_i \times_Q C_i=  R_{i i}$, donc le diagramme
$$
    \xymatrix{
        R_{ii} \ar[r]^{p_1} \ar[d]_{p_2} & C_i \ar[d]^\pi\\
        C_i \ar[r]_{\pi} & M_i
    }
$$
est cartésien et cocartésien ; puisque $R_{i i}$ et $C_i$ sont dans $K$, $M_i$ est dans $K$ (par (4)).

Soient $i, j \in I$. Si $R_{i j}$ est vide, alors $M_i \times_Q M_j$ est vide ; si $R_{i j}$ n'est pas vide, les projections
$$
p_1: R_{i j} \to C_i \quad p_2: R_{i j} \to C_j
$$
sont des épimorphismes par (2), donc $M_i$ et $M_j$ définissent le même sous-objet de $Q$. Puisque les $M_i$ sont dans $K$ et recouvrent $Q$, cela prouve que $Q$ est dans $S$, c.q.f.d.

\vskip .3cm
{
Corollaire (de 2.4.2) {\bf 2.4.3.} --- \it 
S'il existe en outre une petite famille $(Y_{\alpha})$ d'objets de $K$ telle que tout objet de $K$ puisse être recouvert (dans $T$) par des $Y_{\alpha}$, alors $S$ est un $\cU$-topos et l'inclusion $S \to T$ définit un morphisme de topos $T \to S$ (par le critère de Giraud).
}

\vskip .3cm
{
Proposition {\bf 2.4.4.} --- \it Soient $L$ et $M$ deux objets de $E$. Tout objet connexe de $E$ qui trivialise $L$ et $M$ trivialise le faisceau $\underline{\Isom}(L, M)$ des germes d'isomorphisme de $L$ dans $M$. (immédiat)
}
\vskip .3cm 

\vskip .3cm
{\bf 2.4.5.}
\begin{enumerate}
    \item[(i)] {Définition. --- \it Soient $G$ un groupe de $E$, $F$ un objet de $E$ et $G \times F \xlongrightarrow{a} F$ une opération de $G$ sur $F$. On dit que $G$ opère \emph{transitivement} sur $F$ si les conditions équivalentes que voici sont remplies~:
    \begin{enumerate}
        \item[1)] Le quotient $F/G$ est un sous-objet de l'objet final (un ouvert).
        \item[2)] Le morphisme $(a, pr_2): G \times F \to F \times F$ est un épimorphisme.
    \end{enumerate}}
    \item[(ii)] {Lemme. --- \it Soit $L$ un objet localement constant de $E$. Si le groupe constant $\Aut(L)_E$ opère transitivement sur $L$, les composantes connexes de $L$ sont des objets \emph{galoisiens} ; et deux composantes connexes de $L$ situées au-dessus de la même composante connexe de $e_E$ sont toujours \emph{isomorphes}.}
\end{enumerate}
\vskip .3cm

{\it Démonstration}. Soit $K$ une composante connexe de $L$.
\begin{enumerate}
    \item[a)] Pour tout $\alpha \in \Aut(L)$, l'image $\alpha (K)$ de $\alpha_{/K}$ est une composante connexe de $L$;
    \item[b)] Soit $H$ le sous-groupe de $\Aut(L)$ formé des automorphismes $\alpha$ tels que $\alpha(K) = K$ comme sous-objets de $L$. D'après (a) on a un diagramme cartésien :
    $$
    \xymatrix{
        H_E \times K \ar@{_{(}->}[d] \ar[r] & K\times K \ar@{_{(}->}[d] \\
        \Aut(L)_E\times L \ar[r] & L\times L
    }
    $$
    et ainsi $H_E$ opère transitivement sur $K$, donc aussi $\Aut(K)_E$. Mais d'après (2.2.6 (b)), $\Aut(K)_E \times K \to K \times K$ est un monomorphisme, donc $K$ est galoisien. Le reste de la proposition est trivial.
\end{enumerate}

\vskip .3cm
{
Proposition {\bf 2.4.6.} --- \it Soit $L$ un objet localement constant de $E$. Au-dessus de chaque composante connexe de $e_E$, il existe un objet galoisien $Y$ qui trivialise $L$ et tel que tout objet connexe de $E$ qui trivialise $L$ trivialise $Y$.
}
\vskip .3cm

{\it Démonstration}. Il suffit de le prouver pour $E$ connexe. Soit $U$ un objet connexe non-vide qui trivialise $L$ et soit $I = \Hom(U, L)$. Représentons le faisceau $\Isom(I_E, L)$ par un objet $T$ de $E$, d'où un $T$-isomorphisme
$$
I_T \isom T \times I_E \to T \times L.
$$
Tout objet de $E$ qui trivialise $L$ trivialise $T$ (2.4.4.) et $T$ est un pseudo-torseur (en fait un torseur 2.2.6. a)) sous le groupe constant des permutations de $I$ ; d'où notre proposition, par 2.4.5 (ii) et 2.2.5 (ii).

\vskip .3cm
{
Corollaire {\bf 2.4.7.} --- \it Pour tout objet $X$ de $\SLC(E)$, il existe des objets galoisiens $(Y_{\alpha})$ de $E$ et une famille épimorphique $(Y_{\alpha} \to X)$ de $E$.
}

\vskip .3cm
{
Proposition {\bf 2.4.8.} --- \it Soient $Y$ un objet connexe de $E$ et $U$ l'image de $Y \to e_E$. Pour que $Y$ soit galoisien, il faut et il suffit qu'il se trivialise lui-même et que $U$ soit une composante connexe de $e_E$.
}
\vskip .3cm

{\it Démonstration}. On sait déjà que ces conditions sont nécessaires (2.3.1, Remarque (i)). Prouvons qu'elles suffisent. Soit $V$ un sous-objet de $e_E$ tel que $U \amalg V \isom e_E$. $Y$ et $V$ trivialisent $Y$, donc $Y$ est localement constant ; puisque en outre $Y$ est connexe, il existe un objet galoisien $Z$ au-dessus de $U$, qui trivialise $Y$ et est trivialisé par $Y$ (2.4.6). Le produit $Y \times Z$ n'est pas vide puisque $U$ ne l'est pas, donc $Y$ et $Z$ sont isomorphes (2.3.2), c.q.f.d.

\vskip .3cm
{
Proposition {\bf 2.4.9.} --- \it La catégorie des objets galoisiens de $E$ est $\cU$-petite à équivalence près.
}
\vskip .3cm

{\it Démonstration}. Compte tenu de l'hypothèse sur l'univers $\cU$, cela résulte des deux points suivants :
\begin{enumerate}
    \item[(i)] Soient $S$ une catégorie $\cU$-petite et $I$ un ensemble $\cU$-petit. L'ensemble des préfaisceaux $F$ sur $S$ tels que $F(X) \subset I$ pour tout $X \in \Ob(S)$ est $\cU$-petit. (Clair)
    \item[(ii)] Soient $G \subset \Ob(E)$ une petite famille génératrice de $E$, formée d'objets connexes non-vides. Soient $c$ le cardinal de l'ensemble des flèches entre objets $\in G$, et $d$ le cardinal dénombrable. Pour tout objet galoisien $Y$ de $E$ et tout $U \in \Ob(G)$, on a :
    $$
    \Card(\Hom(U, Y)) \leq c^d.
    $$
    
    Démonstration. Soit $A$ l'ensemble des flèches $U \to Y$, $U$ parcourant $G$. Étant donné un élément $U \xlongrightarrow{s} Y$ de $A$, l'ensemble des $(V \xlongrightarrow{t} Y) \in A$ tels que $U \times_Y V$ soit non-vide a son cardinal au plus égal à $c^2$ :
    
    Étant données des flèches $W \to U$, $W \to V$ avec $V$, $W \in G$, il existe au plus une flèche $t: V \to Y$ qui rend commutatif le diagramme
    $$
    \xymatrix{
        W\ar[r] \ar[d] & V \ar[d]^t\\
        U \ar[r]_s & Y
    }
    $$
    (en effet, ou bien $\Hom(V, Y) = \emptyset$, ou bien $V$ trivialise $Y$ et alors $\Hom(V, Y) \to \Hom(W, Y)$ est bijective). Puisque l'ensemble des flèches appartenant à $A$ recouvre $Y$, on déduit de (1.3) que :
    $$
    \Card(A) \leq \sum_{n \geq 0} c^{2n} \leq c^d
    $$
    d'où a fortiori notre assertion.
\end{enumerate}

\vskip .3cm
{
Proposition {\bf 2.4.10.} --- \it Soit $u: A \to B$ un morphisme de topos tel que le foncteur $u^{-1}$ soit \emph{pleinement fidèle}.
\begin{enumerate}
    \item[(i)] Pour qu'un objet $Y$ de $B$ soit connexe et non-vide, il faut et il suffit que $u^{-1}(Y)$ le soit.
    \item[(ii)] Supposons $B$ localement connexe. Pour qu'un objet $Y$ de $B$ soit galoisien, il faut et il suffit que $u^{-1}(Y)$ le soit.
\end{enumerate}
}
\vskip .3cm

{\it Démonstration}.
\begin{enumerate}
    \item[(i)] On a pour tout ensemble $I$ des bijections naturelles
    $$
    \Hom(Y, I_B) \isommap \Hom(u^{-1}(Y), u^{-1}(I_B)) \isommap \Hom(u^{-1}(Y), I_A). \quad \text{(cf 1.2)}
    $$
    \item[(ii)]
    \begin{enumerate}
        \item[a)] L foncteur $u_*$ commute aux sommes directes : en effet, pour tout objet connexe non-vide $D$ de $B$, le foncteur $\Hom_A(u^{-1}(D), -)$ y commute (i). 
        \item[b)] Supposons $u^{-1}(Y)$ galoisien. Soit $V$ l'image de $Y \to e_B$. $u^{-1}(V)$ est connexe et non-vide, et il existe un objet $U'$ de $A$ tel que 
        $$
        u^{-1}(V) \amalg U' \isom e_A
        $$
        on a alors (d'après (a))
        $$
        u_* u^{-1}(V) \amalg u_*(U') \isom u_*(e_A) \isom e_B.
        $$
        Or $u_*u^{-1}(V)$ est isomorphe à $V$ puisque $u^{-1}$ est pleinement fidèle. Donc $V$ est une composante connexe de $e_B$ d'après (i).
        
        $Y$ se trivialise lui-même puisque $u^{-1}$ se trivialise lui-même et que $u^{-1}$ est pleinement fidèle ; $Y$ est connexe et non-vide d'après (i). Donc $Y$ est galoisien d'après 2.4.8.
        \item[c)] Réciproquement, supposons $Y$ galoisien. $u^{-1}(Y)$ est alors localement constant, connexe et non-vide (i) ; et puisque le morphisme de groupes $\Aut(Y) \to \Aut(u^{-1}(Y))$ donné par le foncteur $u^{-1}$ est un isomorphisme, $u^{-1}$ est un pseudo-torseur sous le groupe constant $\Aut(u^{-1}(Y))_A$.
    \end{enumerate}
\end{enumerate}

\vskip .3cm
{\bf 2.4.11.} Le théorème 2.4 est maintenant prouvé : 

Le point (i) par 2.4.1., 2.4.7., 2.4.9. et le critère de Giraud ;

D'après 2.4.7. et 2.4.10, (i), $\SLC(E)$ est localement connexe ; d'où le point (ii) par 2.4.10 (ii).

\vskip .3cm
{
Proposition {\bf 2.4.12.} --- \it Soient $F$ un second $\cU$-topos localement connexe et $E \to F$ un morphisme. Le foncteur image inverse transforme les objets de $\SLC(F)$ en objets de $\SLC(E)$, d'où un morphisme de topos $\SLC(E) \to \SLC(F)$.
}
