%%%%%%%%%%%%%%%%%%%%%%%%%%%%%%%%%%%%%%%%%%%%%%%%%%%%%%%%%%%%%%%
\chapter*{\S \space I. --- OBJETS CONNEXES DANS UN TOPOS}\thispagestyle{empty}
\label{sec:1}
\section*{}

Tous les topos considérés sont des $\cU$-topos.

\vskip .3cm
{
Définitions {\bf 1.1.} --- \it
\begin{enumerate}
    \item[a)] Un objet d'un topos est \emph{connexe} s'il n'est pas somme directe de deux objets non-vides.
    \item[b)] Soit $X$ un objet d'un topos. On appelle \emph{composante connexe} de $X$ tout sous-objet connexe et non-vide $C$ de $X$ tel que $X$ soit somme directe de $C$ et d'un autre objet.
    \item[c)] Un \emph{topos} est connexe si son objet final est connexe.
    \item[d)] Un topos est \emph{localement connexe} s'il est engendré par ses objets connexes. 
\end{enumerate}
}

\vskip .3cm
{
Proposition {\bf 1.2.} --- \it Soit $C$ un objet d'un topos $E$. Les propriétés suivantes sont équivalentes~:
\begin{enumerate}
    \item[a)] $C$ est connexe et non-vide.
    \item[b)] Le foncteur
    $$
    \Hom_E(C, -): E \to \Ens 
    $$
    commute aux sommes directes.
    \item[c)] Pour tout ensemble $I$, l'application naturelle $I \to \Hom(C, I_E)$ est bijective (c'est immédiat).
\end{enumerate}
}

\vskip .3cm
{
Proposition {\bf 1.3.} --- \it Soit $(U_i \xlongrightarrow{f_i} V)_{i \in I}$ une famille épimorphique d'un topos $E$. 

Considérons les propriétés~:
\begin{enumerate}
    \item[(a)] $V$ est connexe et non-vide.
    \item[(b)] Le graphe $R \subset I \times I$ de la relation
    $$
    ``U_i \times_V U_j \quad \text{n'est pas vide''}
    $$
    est connexe (en tant que graphe ayant $I$ pour ensemble de sommets). On a
\end{enumerate}   
\begin{enumerate}
    \item[(i)] si les $U_i$ sont non-vides, (a) entraîne (b).
    \item[(ii)] si les $U_i$ sont connexes, et non-vides, (b) entraîne (a).
\end{enumerate}
}
\vskip .3cm

{\it Démonstration}. C'est trivial si $I = \emptyset$. On suppose donc $I \neq \emptyset$.

{\bf $a \Rightarrow b$} ($U_i$ non-vides). Soit $(I_1, I_2)$ une partition de $I$ telle que pour tout $i \in I_1$ et tout $j \in I_2$, $U_i \times_V U_j$ soit vide. Si on désigne par $V_1$ et $V_2$ respectivement les images des morphismes
$$
\amalg_{i \in I_1} U_i \to V, \quad \amalg_{i \in I_2} U_i \to V
$$
alors $V$ est somme de $V_1$ et $V_2$. Donc $V_1$ ou $V_2$ est vide. Comme aucun
$U_i$ n'est vide, $I_1$ ou $I_2$ est vide.

{\bf $b \Rightarrow a$} ($U_i$ connexes et non-vides). Soit $(Y_{\alpha})_{\alpha \in A}$ une famille d'objets de $E$, et considérons un morphisme
$$
V \to Y = \amalg_{\alpha} Y_{\alpha}.
$$
Pour chaque $\alpha \in A$, soit $I_{\alpha}$ l'ensemble des $i \in I$ tels que le composé
$$
U_i \to V \to Y
$$
se factorise par $Y_{\alpha}$. Puisque les $U_i$ sont connexes et non-vides, $I$ est réunion disjointe des $I_{\alpha}$. Soient $\alpha$ et $\beta$ deux indices distincts. Si $i \in I_{\alpha}$ et $j \in I_{\beta}$, $U_i \times_V U_j$ est vide puisque c'est un sous-objet de $U_i \times_Y U_j$. Appliquant (b), on voit que $I = I_{\alpha_0}$ pour un $\alpha_0 \in A$ et un seul. Donc $V$ est connexe et non-vide par $(1.2).$

\vskip .3cm
{
Proposition {\bf 1.4.} --- \it Tout objet d'un topos localement connexe est somme directe d'objets connexes (donc somme directe de ses composantes connexes).
}
\vskip .3cm

{\it Démonstration}. Soit $E$ un topos localement connexe. Soient $Y$ un objet de $E$ et $(U_i \to Y)_{i \in I}$ une famille épimorphique de $E$, où les $U_i$ sont connexes et non-vides. Soit $R$ le graphe de la relation $``U_i \times U_j$ n'est pas vide''. Pour chaque composante connexe $r$ de $R$, soit $C_r$ l'image dans $Y$ de la somme des $U_i$, $i$ parcourant l'ensemble des $i \in I$ qui sont sommets de $r$. $Y$ est somme directe des $C_Y$, qui sont connexes et non-vides d'après (1.3).

\vskip .3cm
{
Proposition {\bf 1.5.} --- \it Pour qu'un topos $E$ soit localement connexe, il faut et il suffit que le foncteur
\begin{align*}
  \Ens \to E\\
  I \mapsto I_E
\end{align*}
admette un adjoint à gauche
$$
c: E \to \Ens.
$$

Dans ce cas, étant donné un objet $X$ de $E$, les produits fibrés
	\[
		\xymatrix{
			X_\gamma \ar[r] \ar[d]  & e_E \ar[d]^{\gamma} \\
			X\ar[r]                       & c(X)_E
		}
	\]
($\gamma$ parcourant $c(X)$) sont les composantes de $X$.
}
\vskip .3cm

{\it Démonstration}.
\begin{enumerate}
    \item[(i)] Supposons $E$ localement connexe. Pour tout objet $X$ de $E$, désignons par $c(X)$ l'ensemble des classes de $X$-isomorphisme de composantes connexes de $X$ (cet ensemble est bien sûr $\cU$-petit). Soit $f: X \to Y$ un morphisme de $E$ ; étant donnée une composante connexe $C$ de $X$, il existe une composante connexe $D$ de $Y$, unique à $Y$-isomorphisme près, telle que $f_{/C}$ se factorise par $D$. D'où une application
    $$
    c(X) \to c(Y).
    $$
    On a ainsi obtenu un foncteur covariant
    $$
    c: E \to \Ens.
    $$
    Le foncteur $c$ est adjoint à gauche de $I \mapsto I_E$: en effet, étant donnés un objet $X$ de $E$ et un ensemble $I$, on définit une application : 
    $$
    \App(c(X), I) \to \Hom(X, I_E)
    $$
    en associant à l'application
    $$
    a: c(X) \to I
    $$
    le morphisme $X \to I_E$ dont la restriction à chaque composante connexe $C$ de $X$ est la section de $I_E$ au-dessus de $C$ définie par $a(C) \in I$ ; cette application est bijective par (1.2), et elle est fonctorielle en $X$ et $I$.
    \item[(ii)] Inversement, supposons qu'on ait un adjoint à gauche $c: E \to \Ens$ du foncteur $I \to I_E$.
    
    Soit $X$ un objet de $E$. Avec les notations de l'énoncé, $X$ est somme directe des $X_{\gamma}$, $\gamma \in c(X)$. Il suffit donc de prouver que les $X_{\gamma}$ sont connexes et non-vides. Or, pour tout ensemble $I$, les applications naturelles
    $$
    I \to \Hom(X_{\gamma}, I_E)
    $$
    fournissent une application
    $$
    I^{c(X)} \to \prod_\gamma \Hom(X_{\gamma}, I_E)
    $$
    qui rend commutatif le diagramme
    \[
        \xymatrix{
        I^{c(X)} \ar[d] \ar[dr]^-\sim \\
        \prod_\gamma \Hom(X_{\gamma}, I_E) \ar[r]^-\sim & \Hom(X, I_E)
        }
    \]
    donc chacune des applications
    $$
    I \to \Hom(X_{\gamma}, I_E)
    $$
    est bijective ; on conclut par (1.2).
\end{enumerate}
